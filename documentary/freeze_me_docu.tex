\documentclass[12pt]{scrartcl}
\usepackage[german]{babel}
\usepackage{graphicx}
\usepackage[latin1]{inputenc}
\usepackage[T1]{fontenc}

\usepackage{color}
\usepackage[linkcolor=black, urlcolor=black, citecolor=dblue,
breaklinks, bookmarks, colorlinks]{hyperref}

\begin {document}

\hspace{-0.4cm}\huge \textbf{Freeze Me! - Dokumentation}\vspace{3pt}\\\Large A Media Processing Project \vspace{14pt}\large 

\noindent
\textbf{Stand:} Tag der Abgabe! \\ 

\vspace{3pt} \normalsize Gruppe: 
Jan Rekemeyer, Iskander Yusupov und Hendrik Finke

\section{Einleitung und Ziel} %Iskander wollte hier noch etwas schreiben
\subsection{Motivation}
Da das Ziel des Projekts darin bestand, ein Softwareprodukt zu erstellen, entschied sich die Gruppe, eine Motivation im Marketingformat zu erstellen.\\Gl\"uckliche Momente lassen sich in einem Foto festhalten, das dann verschenkt werden kann. Aber was tun mit den gl\"ucklichen Momenten, die auf Video festgehalten wurden? Mit diesem Ansatz suchte die Gruppe nach M\"oglichkeiten, ein Video so in Form eines Bildes festzuhalten, dass dieses sp\"ater beispielsweise als gerahmtes Geschenk pr\"asentiert werden k\"onnte.
Eine weitere Inspiration f\"ur unsere Gruppe waren Videos mit viel Dynamik. Sport, Tanz, aber auch Tiktok-Challenges enthalten viel Bewegung. Beim Fotografieren von sich bewegenden Objekten erzeugt die Einstellung einer Langzeitbeleuchtung den Effekt einer \glqq eingefrorenen\grqq  Bewegung. Diese Fotos haben ihre eigene \"Asthetik und unsere Gruppe wollte versuchen, diesen Effekt mit den im Modul erlernten F\"ahigkeiten zu reproduzieren.
\subsection{Zielsetzung}Unser Projekt haben wir der folgende Zielsetzung gewidmet: \\
\textit{Wir entwickeln ein Programm, welches es erm\"oglicht aus einem Video ein Bild zu erzeugen, welches die Bewegungsdynamik des Videos \"asthetisch einfasst und dabei den Eindruck fortw\"ahrender Bewegung erh\"alt.}

\section{Inspiration: Langzeitbelichtung} %Erstmal Hendriks Sektion
Das in den Er\"offnungsfolien gezeigte Bildmaterial erinnerte uns in Teilen doch recht stark an eine klassische Langzeitbelichtung, wie man sie aus der Analogfotografie oder der Fotographie mit DSLRs oder DSLMs kennt. Daher stellten wir uns die Frage, wie man dieses Konzept digital nachtr\"aglich unter Zuhilfenahme eines Videos anstelle eines Fotos reproduzieren kann. \\
Der Ansatz, den wir hierbei Verfolgt haben ist relativ trivial. \"Ahnlich wie klassiche Filter den Inhaltswert eines Pixels anhand einer ggf. faktorisierten Durchschnittsberechnung mit den umliegenden Pixeln berechnet tun wir ebendieses - jedoch mit den Pixeln gleicher Koordinaten auf anderen Videoframes. 

\section{Edge Detection und Background Subtraction}

\end{document}
